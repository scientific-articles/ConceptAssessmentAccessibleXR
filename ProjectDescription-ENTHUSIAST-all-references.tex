
\subsubsection{State of the art}

\subsubsection{Knowledge needs and project objectives}
Participatory design and co-creation are considered to be some of the best methods for understanding a target audience
and creating high quality, well designed solutions to meet their needs and preferences. Despite this, several reviews
conclude that the user-involvement in the design and development of e-health services is
insufficient\footnote{Sosial- og helsedepartementet (2001). Fra bruker til borger. En strategi for nedbygging av
funksjonshemmende barrierer. Oslo.}, \autocite{manzoorEservicesSocialInclusion2017a}, and needs to be strengthened\footnote{\ Forskningsrådet. Evaluering av
samhandlingsreformen: sluttrapport fra styringsgruppen for forskningsbasert følgeevaluering av samhandlingsreformen
(EVASAM). Oslo: Forskningsrådet; 2016.}. Moreover, research that develop and evaluate digital health services designed
for people with disabilities, seldom include people with disabilities in the study sample \autocite{10.3389/fpsyg.2018.02323,10.1007/978-3-319-58634-2_15}.

Despite the benefits documented in the scientific literature of XR developed for special user groups, such assistive
technology (AT) has very low uptake and high abandonment rates \autocite{schererWhyPeopleUse2015}.
Thus, assistive technology targeted at small user groups has a very limited potential for increasing equity and
reducing exclusion of vulnerable user groups. Some important reasons for the low uptake and high abandonment rates
could be that a) people with disabilities are often omitted from the research and development process b) the lack of
collaboration between end-users, researchers, and developers in industry c) the lack of policy platforms to support the
translation of prototype conceptualization in research to actual implementation in the community and d) lack of
sustainable business models when the target user groups are small. Also, while studies do report positive effects for
special user groups, it is difficult to generalise from a limited number of participants to larger groups. 

This project will develop a framework with accompanying tools that can be used to evaluate whether XR installations
(hardware and software) are accessible, usable, and suitable for users with various needs and abilities, and to put
forward requirements for such technology. The framework and tools can be used by organisations that acquire or develop
VR/AR to integrate accessibility as early as possible and thus contribute to equal opportunities for people and larger
audiences for VR/AR solutions.

In the second step we will, in cooperation with organisations and industry, create a reference architecture and find and
adapt accessibility plugins, that can be integrated in existing or new XR technology to make them more accessible. 

When people use XR technology, they are presented with a scenario (i.e., a sequence of events in an
environment)\footnote{Simulation Interoperability Standards Organization. SISO-GUIDE-006-2018 – Guideline on
Scenario Development for Simulation Environments, 2018; Siegfried R, Laux A, Rother M et al. Scenarios in
military (distributed) simulation environments. In Proc. 2012 Spring Simulation Interoperability Workshop.
12S-SIW-014.}. The basic idea is that scenarios should be described digitally in a machine-readable format.
Furthermore, there should be enough information in this description for the scenario to be presented in different
modalities that can be adapted to people with different abilities. The digital scenario description is thus universally
designed, while individual adaptations are made at the presentation level. 

This requires a loose coupling between the scenario as such and how the scenario is implemented in various mixed reality
platforms (Unity, Unreal, VBS, etc.). The project will develop a reference architecture\footnote{The Open Group: SOA
Reference Architecture Technical Standard (2011), doc. no. C119} \autocite{hannay-vandenBerg-NATO-2017}
for universally designed digital scenarios as described
above. A reference architecture is a design template, which systems developers can use when implementing solutions.
Adhering to this reference architecture should ensure that solutions embody the stated principles for universal design
and adaptations to particular user groups. Figure 1 outlines the idea of this reference architecture. The project will
focus on what information digital scenarios must contain to be universally designed (green artefact in the middle), as
well as to find out how the information can be used in different mixed-reality platforms to implement adaptations to
different users (green boxes on the right). The balance between generic and customization-specific information will be
a central theme.

Figure 1 \WVL{[tbd.]} also shows how a system that uses universally designed digital scenarios is intended to be used. Healthcare
professionals should be able to design scenarios in an easy-to-understand tool 
\autocite{10.1007/978-3-030-50732-9_60,Hannay2019StructuredCT}, 
and a scenario generator will represent the scenario digitally, together with any scoring models to
measure relevant aspects during plays \autocite{DashleyK.RouwendalvanSchijndel_etal2020}. 
When the scenario is run on different mixed reality
platforms with different adaptations, events and changes in the environment is largely determined by a joint event
manager and a stage \& content manager, respectively (ibid). 

\subsection{Research questions and hypotheses, theoretical approach and methodology}
Main hypotheses: 

\begin{enumerate}
\item It is possible to create a framework with accompanying tools that will enable developers to create more
accessible, inclusive, universally designed mainstream XR solutions. These created solutions will reach beyond the
target groups of users with disabilities. We will reach additional target groups like for example the elderly or people
who are situationally disabled by employing more flexible and adaptable techniques. Finally, we will contribute to
reduce inequality of XR-based health and welfare services. 
\item When presenting user needs in a systematic form, the goals and purposes of universal design of XR, together with
concrete ideas and solutions for specific accessibility challenges, and offer empirical evaluation and testing with end
users together with partners in the health sector and industry will seize the opportunity and implement accessibility
features into some of their solutions.
\end{enumerate}
\subsubsection[\ Research questions: ]{\ Research questions: }
\begin{itemize}
\item What type of co-creation methods can be used to enhance the voice of users and stakeholders in the different
stages of the problem-solving process? 

\begin{itemize}
\item What type of XR are considered important/interesting for Enthusiast target users. 
\item What features are important for accessibility and user experience? 
\item When is XR appropriate for Enthusiast target users? When is it not?
\item What are the benefits and limitations of XR in health and welfare related training scenarios for our target
groups?
\end{itemize}
\item What are the characterizing factors of universally designed XR solutions and how can the degree of accessibility
and engagement for various user groups be described and measured? 

\begin{itemize}
\item What criteria are important for an inclusive user experience of XR?
\item What other frameworks can be used as a basis for developing our framework (universal design for learning,
engagement profiles, universal design in other contexts, e.g. museums)?
\end{itemize}
\end{itemize}
\begin{itemize}
\item \begin{itemize}
\item How can we measure accessibility and user experience? (e.g. through observations, measurements, etc.? Can sensors,
Brain Computing Interface (BCI) or other tools be used for automatic evaluation of the user experience – as a
supplement to empirical evaluation methods? \ 
\item How can our framework and accessibility be used by software/hardware producers and their customers to create more
inclusive XR systems? What are concrete recommendations for hardware manufacturers and software developers.
\end{itemize}
\end{itemize}
\begin{itemize}
\item How can we define a reference architecture that allows XR manufacturers to integrate accessibility plugins into
their mainstream XR? 

\begin{itemize}
\item Which aspects should be covered by the universally designed digital scenario, and which aspects should be covered
by the specific adaptations to different user groups (Figure 1)? For example, should engagement principles be stated as
part of the universally designed scenario or as part of the adaptations. 
\item How well does the reference architecture accommodate state-of-practice and possibly proprietary stove-piped
solutions; hereunder,

\begin{itemize}
\item What types of accessibility objects/plugins exists and how are they integrated with \ \ mainstream XR? 
\item How feasible is it that industry will adhere to the reference architecture ideal? 
\end{itemize}
\item What is the user experience of users with disabilities of the adapted solution? 
\end{itemize}
\end{itemize}
\begin{itemize}
\item How can municipal, industry and other stakeholder be educated and engaged in the promotion of universal design of
XR? 
\end{itemize}
\subsubsection{Theoretical approach and methodology}
Co-creation describes an interaction where actors jointly produce a mutually valued outcome based on their assessments
of the risks and benefits of the proposed courses of action, and decisions based on dialogue, access to information and
resources, as well as transparency\footnote{\ Prahalad, C. K., \& Ramaswamy, V. (2004). Co-creation experiences: The
next practice in value creation. Journal of Interactive Marketing, 18(3), 5-14. doi:https://doi.org/10.1002/dir.20015
}. Co-creation has the potential to reduce siloed-working. It may refer to any act of collective creativity, i.e.
creativity that is shared by two or more people\footnote{Sanders, E.B.N. \& P.J. Stappers. (2008) Co-creation and the
new landscapes of design. CoDesign. 4(1):5-18.}. Co-creational approaches including actively involving intended users
has increasingly been applied to redesigning health services,\footnote{Chamberlain, P. \& R. Partridge
(2017). Co-designing co-design. Shifting the culture of practice in healthcare. The Design
Journal. 20 (sup. 1):2010-2021}. In this project we define co-creation pragmatically as creative, open and participatory
and inclusive processes aiming at a) identifying user needs, b) exploring, building and sharing knowledge, ideas and
concerns across organisational roles and cultures, and c) making the iterative minor and major adjustments necessary
for reaching the strategic research goals and to develop accompanying and practical tools\footnote{Dugstad, J., et
al. (2018), Towards successful implementation of digital night monitoring technology through co-creation. A four-year
longitudinal case-study}.


\subsubsection{Interdisciplinary consideration and cooperation}
The proposal brings together organisations and individuals in a unique interdisciplinary and cross-sectional
collaboration in rehabilitation research in Norway and beyond. Building on the network of the project partners and the
reference group we will reach out to national and international stakeholders. This will further strengthen the network
between service users in municipality and research in the implementation and use of XR technology in the health and
welfare sector to ensure equity and equality to health and welfare services.

\subsubsection[\ Ethical considerations]{\ Ethical considerations}
Close collaboration with stakeholders will ensure RRI (Responsible research \& innovation). As the project involves
vulnerable groups, ethical aspects will be considered during all stages of the project. Further, ethical considerations
will be systematically integrated into the co-creation and evaluation processes. All participation by stakeholders and
users will be regulated by informed consent. To address ethical aspects related to the use of XR as health technology,
we will use the Morally Relevant Questions with Respect to Assessing Health Technology introduced by 
\citet{hofmann_droste_oortwijn_cleemput_sacchini_2014}
%Hofmann et al.
%(2014)\footnote{\ Hofmann, B., Droste, S., Oortwijn, W., Cleemput, I., \& Sacchini, D. (2014). Harmonization of ethics in health technology assessment: a revision of the Socratic approach. International journal of technology assessment in health care, 30(1), 3.} 
including 33 questions within seven domains. 
What are\dots: 
\begin{enumerate*}[label={\arabic*.\ },ref=\arabic*]
\item \dots the morally relevant issues related
to the diagnosis and the patient group? 
\item \dots the ethical, social, cultural, legal, and religious challenges related to
the health technology? 
\item \dots the moral challenges with structural changes related to the health technology? 
\item \dots the moral
issues related to the characteristics of the health technology? 
\item \dots the moral issues related to stakeholders? 
\item \dots the
moral issues related to the assessment of the health technology? 
\item Are there additional moral issues?
\end{enumerate*}

\subsection[Novelty and ambition ]{Novelty and ambition }
That the universally designed scenarios do not have to be implemented time and time again for each new adaptation, and
that adaptations are made only in the mixed-reality platforms, is a loose coupling in line with the concept Modelling
\& Simulation as Service (MSaaS\footnote{\ van den Berg, T.W., Huiskamp, W., Siegfried, R., Lloyd, J., Grom, A.,
Phillips, R.: Modelling and Simulation as a Service: Rapid deployment of interoperable and credible simulation
environments – an overview of NATO MSG-136. In: Proc. 2018 Winter Simulation Innovation Workshop. No. 18W-SIW-018
(2018)}[FFFC?] . This concept is essential for enabling universally designed solutions for different user groups
quickly. The flexibility in this is also necessary for rapid testing and evaluation of solutions.

The anticipated results are both theoretical and practical:
\begin{enumerate}
\item 
A survey and overview of challenges and opportunities of VR for people with disabilities. 
This milestone has as overall goal to raise awareness and deepen the understanding of accessibility in VR.

\item An assessment tool to evaluate the accessibility of VR systems, hardware, software, etc. with respect to the
dimensions of cognition, mobility, hearing, vision, motor, and/or voice. In this milestone, we will provide a
checklist or some other half-automated tool that can be used by manufacturers, developers and customers of VR systems
to evaluate the accessibility of any given VR system.

\item A survey of existing tools and solutions to address some of the challenges discovered in the previous milestones. We
will list and review plugins, hardware, development frameworks, and provide some suggestions on how to compensate for
some of the challenges that people with impairment(s) might face.

\item A specification on how to make VR systems more accessible for people with disabilities.
The outcome of this milestone will be a list of recommendations and concrete specifications that manufacturers and
developers can use to create well-rounded and universally designed VR system.

\item An empirical evaluation with actual user groups connected to our partners (most likely the deaf and/or people with
autism).
We will verify the ENTHUSIAST framework empirically with one/two concrete user groups: from the assessment of existing
games and setups, suggestions on how to improve them, implementations of our specifications (in form of plugins,
add-ons, extra-hardware, etc), to the evaluation of accessibility and perceived engagement before and after
accessibility improvement.
\end{enumerate}

\section[Impact ]{Impact }
\subsection[Potential impact of the proposed research ]{Potential impact of the proposed research }
There are several outputs from the project. First there is the UD-XR evaluation and development framework. This will
describe accessibility needs and requirements for a broad range of user groups, and with detail and emphasis on the
needs of the user groups in our partner organisations. Secondly, there will a solution architecture with a set of
accessibility features that can be implemented. 

With the advancing commercial feasibility, ongoing popularity and omnipresent digitalisation of public and official
life, it is very probable that there are going to be more legal requirements to digital accessibility which includes
VR/AR/MR21, similar to for example the 2019 European Accessibility
Act\footnote{https://eur-lex.europa.eu/legal-content/EN/TXT/?uri=uriserv:OJ.L\_.2019.151.01.0070.01.ENG\&toc=OJ:L:2019:151:TOC
\ } and the Norwegian Equality and Anti-Discrimination Act\footnote{https://lovdata.no/dokument/NLE/lov/2017-06-16-51}.
The UN Convention on the Rights of Persons with Disabilities aims to ensure that disabled people can enjoy the full
range of human rights: civil, political, economic, social, and cultural. This shall, among other things, be ensured
through requirements for accessible ICT (Article 2). The Convention refers to the concept of universal design (UD) as a
means through which to achieve this goal (Article 4). 

As in the web accessibility domain there will be need for assistive technology (AT), specially designed for certain user
groups. However, integrating accessibility as a fundamental core part of a mainstream XR-system will be valuable not
only for people with disabilities, but also to create a more flexible, adaptive and inclusive technology ecosystem that
will benefit a wide range of users, such as people of different ages, languages, and everyone experiences situational
disabilities dependent on their context21. Universally designed mainstream technology has many advantages over
specialized technology, such as lower cost, increased availability, and creating equal opportunities and social
acceptability. 



